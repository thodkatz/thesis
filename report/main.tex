% !TEX program = xelatex

\documentclass[12pt, a4paper]{article}

\usepackage{fontspec}
\setmainfont[Ligatures=TeX]{Linux Libertine O}
\usepackage[left=2.5cm,right=2.5cm,top=2.5cm,bottom=2.5cm]{geometry}
\usepackage{amsmath,amsfonts,amssymb}
\usepackage{dirtytalk}
\usepackage{bookmark}
\usepackage{cite}
\usepackage{graphicx}
\usepackage{subcaption}
\usepackage{float}
\usepackage{siunitx}
\usepackage{color}
\usepackage{indentfirst}

\graphicspath{{assets/}}

\sloppy


\begin{document}

\begin{titlepage}
    \begin{figure}[H]
      \begin{center}
        \includegraphics[width=3cm]{auth.pdf}
        \label{fig:cover_auth_logo}
      \end{center}
    \end{figure}
    
    \centering
    \Large Aristotle University of Thessaloniki\\
    \Large Faculty of Engineering\\
    \large School of Electrical and Computer Engineering\\
    \large Department of Electronics and Computer Engineering

    
    \vspace{\fill}


    \Large Diploma Thesis

    \vspace{\fill}
    
    \Large \textbf{Multi-task learning in perturbation modeling}
    
    \Large \textbf{with application in single-cell data}
    
    \vspace{\fill}
    
    \Large Theodoros Katzalis

    \large School Registration Number: 9282
    
    \vspace{\fill}
    \raggedright
    
    \begin{large}
    \begin{tabular}{ll}
    & \textbf{Supervisor:} \\
    & Pericles Mitkas \\
    & Professor at Aristotle University of Thessaloniki (AUTh) \\
    & \\

    & \textbf{Co-supervisor:} \\
    & Fotis E. Psomopoulos \\
    & Senior Researcher at the Institute of Applied Biosciences (INAB) of the\\
    & Center for Research and Technology Hellas (CERTH) \\

    \end{tabular}
    \end{large}
    
    \centering
    \vspace{\fill}
    \large Thessaloniki, December 2025    
    
    \end{titlepage}
    
    % \begin{abstract}
    % abstract
    % \end{abstract}
    
    % \begin{abstract}
    % abstract
    % \end{abstract}
    
    % \thispagestyle{empty}
    
    
    % \section*{Ευχαριστίες}
    % \thispagestyle{empty}
    
    
    
    \clearpage
    

% \title{Τίτλος διπλωματικής}
% \author{Όνομα Επίθετο \\
% \href{mailto:empty@auth.gr}{empty@auth.gr}}
% \maketitle

{
\renewcommand*\contentsname{Περιεχόμενα}
\hypersetup{linkcolor=black}
\tableofcontents
}

\thispagestyle{empty}

\clearpage

\section{Abstract}

With the recent advancements in single cell technology and the large scale perturbation datasets, the field of perturbation modeling  has created an opportunity for a wide variety of computational methods to be leveraged to harness its potential. Multi-task learning is one of the methods that has been left unexplored in this field. In this study we aim to bridge this gap unraveling the potential of multi-task learning in single cell perturbation modeling.


\section{Introduction}

The complexity of biological systems have imposed a challenge to capture the underlying mechanism of cellular heterogeneity. Understanding the effect of external stimuli (perturbations) to the cell level, a field named as perturbation modeling \cite{jiMachineLearningPerturbational2021}, has a significant impact in biomedicine and drug discovery. With the recent surge of data generation, machine learning methods aim to understand the effect of perturbations, given a limited number of perturbation experiments. 

An overview of the models on perturbation modeling can be found on this study \cite{gavriilidisMinireviewPerturbationModelling2024}. One of the main objectives is the out-of-distribution detection, which is the focal point of our study. The task is about predicting the perturbation response of the omics signature of cells with a specific cell type, while having observed the perturbation response of other cell types.

% One of the key problems of deep learning methods is the data demand. We assume that the utilization of the data of multiple perturbations, under a multi-task context, to be beneficial to mitigate this.

%UnitedNet \cite{tangExplainableMultitaskLearning2023}, an explainable multi-modal framework, had shown the potential of multi-task learning in a multi-omics dataset. We aim to extend this approach to perturbation modeling.

UnitedNet is a multi-task framework that has shown its potential in multi-omics tasks such as cross modal prediction and cell type classification. We aim to extend this approach to perturbation modeling.


\section{Method}

A short intro of multi-task and the rejection of a multi-head architecture. Intro to film layers and explanation of the method.

the usage of film layers

\section{Results and discussion}

In the literature body, there are several approaches for predicting single cell perturbation responses. To compare our multi-task model, we have chosen the models of scGen, scButterfly, scPregan, and scVIDR.

% write a few sentences for each architecture.

scGen used a vae that captures the perturbation response using vectors in the perturbation space

scbutterfly with a vae had shown its potential on the perturbation modeling applied on the pbmc dataset. It is based on vae architecture with these characteristics.

scPreGAN.

scVIDR.

To quantify the objective of the task, we have used a multi-faceted benchmarking suite, relying on distance metrics along with the number of differentially expressed genes (DEGs) and the r2 of all genes and the most highly variable genes.

We have tested the models on two datasets, one where human peripheral blood mononuclear cells have been stimulated by IFN-b interferon, and a multi-perturbation dataset, where liver cells have been stimulated by multiple doses of tetrachlorodibenzo-p-dioxin (TCDD) in vivo.

Regarding the single perturbation response models, the scGen, scButterfly, scPreGAN, in the multi-perturbation dataset of ten dosages, we have trained a dedicated model for each dosage.

To address the randomness of the models, we have performed the experiments three times, with thres different seeds 1, 2, 19193, and the metrics have been averaged across experiments.


\section{Conclusions}

\section{Future work}


\chapter{Benchmarking}
\label{ch:chapter1}


\section{Datasets}



\appendix
\chapter{Ακρωνύμια και συντομογραφίες}

\begin{description}
  \item[LAN] Local Area Network
\end{description}



\bibliographystyle{plain}
\bibliography{references.bib}

\end{document}