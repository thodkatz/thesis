% !TEX program = xelatex

\documentclass[12pt, a4paper]{report}

\usepackage{fontspec}
\setmainfont[Ligatures=TeX]{Linux Libertine O}
\usepackage[left=2.5cm,right=2.5cm,top=2.5cm,bottom=2.5cm]{geometry}
\usepackage{amsmath,amsfonts,amssymb}
\usepackage{dirtytalk}
\usepackage{bookmark}
\usepackage{cite}
\usepackage{graphicx}
\usepackage{subcaption}
\usepackage{float}
\usepackage{siunitx}
\usepackage{color}
\usepackage{indentfirst}

\graphicspath{{assets/}}

\sloppy


\begin{document}

\begin{titlepage}
    \begin{figure}[H]
      \begin{center}
        \includegraphics[width=3cm]{auth.pdf}
        \label{fig:cover_auth_logo}
      \end{center}
    \end{figure}
    
    \centering
    \Large Aristotle University of Thessaloniki\\
    \Large Faculty of Engineering\\
    \large School of Electrical and Computer Engineering\\
    \large Department of Electronics and Computer Engineering

    
    \vspace{\fill}


    \Large Diploma Thesis

    \vspace{\fill}
    
    \Large \textbf{Multi-task learning in perturbation modeling}
    
    \Large \textbf{with application in single-cell data}
    
    \vspace{\fill}
    
    \Large Theodoros Katzalis

    \large School Registration Number: 9282
    
    \vspace{\fill}
    \raggedright
    
    \begin{large}
    \begin{tabular}{ll}
    & \textbf{Supervisor:} \\
    & Pericles Mitkas \\
    & Professor at Aristotle University of Thessaloniki (AUTh) \\
    & \\

    & \textbf{Co-supervisor:} \\
    & Fotis E. Psomopoulos \\
    & Senior Researcher at the Institute of Applied Biosciences (INAB) of the\\
    & Center for Research and Technology Hellas (CERTH) \\

    \end{tabular}
    \end{large}
    
    \centering
    \vspace{\fill}
    \large Thessaloniki, December 2025    
    
    \end{titlepage}
    
    % \begin{abstract}
    % abstract
    % \end{abstract}
    
    % \begin{abstract}
    % abstract
    % \end{abstract}
    
    % \thispagestyle{empty}
    
    
    % \section*{Ευχαριστίες}
    % \thispagestyle{empty}
    
    
    
    \clearpage
    

\title{Τίτλος διπλωματικής}
\author{Όνομα Επίθετο \\
\href{mailto:empty@auth.gr}{empty@auth.gr}}
\maketitle

{
\renewcommand*\contentsname{Περιεχόμενα}
\hypersetup{linkcolor=black}
\tableofcontents
}

\thispagestyle{empty}

\clearpage

\section{Abstract}

With the recent advancements in single cell technology and the large scale perturbation datasets, the field of perturbation modeling \cite{jiMachineLearningPerturbational2021} has created an opportunity for a wide variety of computational methods to be leveraged to harness its potential. An overview of the methods that have been tested can be found on this study (cite the mini review). Multi-task learning is one of the methods that has been left unexplored in this field. In this study we aim to bridge this gap unraveling the potential of multi-task learning in single cell perturbation modeling.

\section{Literature review}


Among the tasks within the perturbation modeling, we will focus on the out-of-distribution prediction task. Explain the task.

%Studying the effect of external stimuli to the cell level, a field named as perturbation modeling, has a significant impact to the biomedicine sector and drug discovery. 

A review has been conducted to outline the methods and the datasets. One of the major tasks of perturbation modeling is the out of distribution prediction. 


UnitedNet has followed a multi task approach in a multi-omics datasets, showed the promising results of the method. scButterfly followed a cross modal architecture has proporsed its potential to perturbation modeling.



scbutterfly, scgen, scvidr, unitednet

\section{Method}

the usage of film layers

\section{Results and discussion}

\section{Conclusions}

\section{Future work}


\chapter{Benchmarking}
\label{ch:chapter1}


\section{Datasets}



\appendix
\chapter{Ακρωνύμια και συντομογραφίες}

\begin{description}
  \item[LAN] Local Area Network
\end{description}



\bibliographystyle{plain}
\bibliography{references.bib}

\end{document}